\documentclass[14pt, a4paper]{extreport}
	\usepackage[left=25mm, right=10mm, top=20mm, bottom=20mm]{geometry}
	\usepackage[hidelinks, linktoc=all, pagebackref]{hyperref}
	\usepackage{indentfirst}
    \usepackage{subcaption}
   	\usepackage[biblabel]{cite}
    \usepackage[font=small,labelfont=bf]{caption}
    \usepackage{abstract}
    \usepackage{mathtools}
    \usepackage[russian]{babel}
    \usepackage{mathrsfs}  
    \usepackage{fontawesome}
    \usepackage{cleveref}
    \usepackage{fancyhdr}
    \usepackage{qrcode}
    \usepackage{enumitem}
    \setlength{\parskip}{0.1cm}   
	\setlength{\parindent}{0.7cm}
	\usepackage{setspace}
	\onehalfspacing
    \usepackage{graphicx} % Used to insert images
    \usepackage{adjustbox} % Used to constrain images to a maximum size 
   	\usepackage{color} % Allow colors to be defined
    \usepackage{amsmath} % Equations
    \usepackage{amssymb} % Equations
    \usepackage[mathletters]{ucs} % Extended unicode (utf-8) support
    \usepackage[utf8x]{inputenc} % Allow utf-8 characters in the tex document
    \usepackage{fancyvrb} % verbatim replacement that allows latex
    \usepackage{grffile} 
    \usepackage{tabularx}
	\usepackage{bbm}
	
	\usepackage{tocloft}
	\renewcommand\cftsecleader{\cftdotfill{\cftdotsep}}
	\renewcommand{\cftchapleader}{\cftdotfill{\cftdotsep}}


	\usepackage{titlesec}
	
	\titleformat{\chapter}
	{\normalfont\fontsize{20}{24}\bfseries}{\thechapter}{1em}{}
	
	
	\titleformat{\section}
	{\normalfont\fontsize{16}{20}\bfseries}{\thesection}{1em}{}
	\titleformat{\subsection}
	{\normalfont\fontsize{14}{16}\bfseries}{\thesubsection}{1em}{}

\newcommand{\diff}{\,\mathrm{d}} 	
 \renewcommand{\figureautorefname}{Fig.}%
\renewcommand*\thesection{\arabic{section}}
\DeclarePairedDelimiter\bra{\langle}{\rvert}
\DeclarePairedDelimiter\ket{\lvert}{\rangle}
\DeclarePairedDelimiterX\braket[2]{\langle}{\rangle}{#1 \delimsize\vert #2}
\newcommand{\rbrkt}[1]{\left( #1 \right)}
\newcommand{\sbrkt}[1]{\left[ #1 \right]}

\renewcommand*{\backreftwosep}{ and~}
\renewcommand*{\backreflastsep}{ and~}
\renewcommand*{\backref}[1]{}
\renewcommand*{\backrefalt}[4]{%
\ifcase #1 %
\relax%
\or
(referenced on p. [#2])%
\else
(referenced on p. [#2])%
\fi
}

\numberwithin{equation}{section}
\setcounter{tocdepth}{3}


\renewcommand*\thesection{\arabic{chapter}.\arabic{section}}
\lhead[\rm\thepage]{\fancyplain{}{\nouppercase{\sl{\rightmark}}}}
\rhead[\fancyplain{}{\nouppercase{\sl{\leftmark}}}]{\rm\thepage}
\chead{}\lfoot{}\rfoot{}\cfoot{}
\pagestyle{plain}


\graphicspath{{Pictures/}}


\begin{document}

\selectlanguage{russian}
\begin{titlepage}
\par 
\vspace*{-2cm}
\begin{center}
Министерство образования и науки Российской Федерации\\
Федеральное государственное автономное образовательное учреждение высшего\\ 
профессионального образования \\
<<Московский физико-технический институт (государственный университет)>> 
\end{center}

\vspace*{0.2cm}
\begin{flushright}
На правах рукописи\\
УДК 539.12
\end{flushright}

\vfill

\begin{center}
Федоров Глеб Петрович

\vspace*{0.5cm}

{\large Моделирование квантового взаимодействия излучения и вещества с использованием массивов сверхпроводниковых искусственных атомов}


\begin{center}
Специальность 03.04.01 ---\\ <<Прикладные математика и физика>>\\
\vspace{0.5cm}
Диссертация на соискание учёной степени \\
кандидата физико-математических наук
\end{center}


\vspace*{2cm}


\begin{flushright}
Научный руководитель\\
д. ф.-м. н., проф.\\
Рязанов Валерий Владимирович
\end{flushright}



\vfill

Москва

2021 г. 
\end {center} 
\end{titlepage}


\tableofcontents

\chapter*{Введение}
\addcontentsline{toc}{chapter}{Введение} 

\section*{Актуальность работы}

История развития сверхпроводниковых искусственных атомов, или кубитов, как инструмента для наблюдения макроскопических квантовых эффектов берет своё начало в 1997 году, когда в группе проф. Накамуры в Японии была впервые показана \cite{nakamura1997spectroscopy} когерентность суперпозиции зарядовых состояний одноэлектронного транзистора. Этот эксперимент стал толчком к развитию новой области физики, интерес к которой обеспечивался как открывшимися возможностями изучать фундаментальные физические явления, так и потенциальной применимостью для квантовых вычислений.

Несмотря на относительно недавнее появление, сверхпроводниковые, или джозефсоновские кубиты, как их также называют, прошли много стадий в своем развитии. Системы, использовавшиеся в первых экспериментах, имели очень низкие времена когерентности: например, в работе \cite{nakamura1999coherent} время затухания Раби-осцилляций в зарядовом кубите составило около 1 нс, в то время как сейчас рекордные времена когерентности составляют порядка 100 микросекунд \cite{kjaergaard2020superconducting}. Наиболее далеко в области квантовых вычислений продвинулись американские учёные, работающие теперь в компаниях Google и IBM. В частности, в 2019 году Google продемонстрировали \cite{arute2019quantum} квантовое превосходство своего процессора из 53 кубитов над мощнейшим существующим классическим компьютером. Однако даже этот результат пока ещё далёк от реальных практических применений, так как количество ошибок, происходящих в устройстве, пока еще очень велико, и решаемая задача была создана искусственно для минимизации чувствительности результата к декогеренции. Как было показано еще Шором в ХХ веке \cite{shor1995scheme}, практические квантовые вычисления потребуют реализации алгоритмов коррекции ошибок, что неминуемо требует существенного увеличения числа физических кубитов, требующихся для обеспечения работы небольшого числа логических. Увеличение числа кубитов более, чем на порядок, непременно натолкнётся на трудности масштабирования контролирующей электроники и криогенной аппаратуры \cite{krinner2019engineering}, алгоритмов калибровки системы \cite{arute2019quantum, kelly2018physical}, а также проектирования самих сверхпроводящих интегральных схем \cite{hutchings2017tunable}. Отсюда следует, что сегодня нельзя назвать даже примерных сроков реализации полезных квантовых алгоритмов \cite{arute2019quantum}.

Однако острый интерес к потенциальным применениям в области квантовых вычислений помог мотивировать исследования со сверхпроводящими кубитами и по другим направлениям \cite{kjaergaard2020superconducting}. В частности, чрезвычайно большое количество экспериментов было проведено в области квантовой электродинамики цепей \cite{blais2020quantum}, вдохновленной нобелевскими исследования Гароша \cite{haroche2013nobel} по стандартной квантовой электродинамике полостей. Впервые сверхпроводниковый искусственный атом был сильно связан (strongly coupled) с квантованным полем в резонаторе в 2004 году в Йельском университете \cite{wallraff2004strong}, что стало первым в истории экспериментальным подтверждением возможности связать одиночную квантовую систему с полем так, чтобы сила связи превысила диссипацию. С развитием технологии производства алюминиевых сверхпроводниковых чипов, открытием новых схем расположения их элементов, а также удешевлением электроники все больше групп в мире стали включаться в работу и вести собственные исследования. В Йельском университете работают проф. Деворе (Michel Devoret) и проф. Шелькопф (Robert Schoelkopf), занимающиеся в основном экспериментами с неклассическими состояниями света в микроволновых резонаторах, которые они готовят, используя связанные с ними искусственные атомы (см. \cite{vlastakis2013deterministically, mirrahimi2014dynamically}). В этой группе также берут начало известные работы компании IBM (см. \cite{jurcevic2020demonstration}), а также стартапа Rigetti, по фамилии одного из защитившегося в Йеле аспирантов \cite{reagor2018demonstration}. В группе проф. Мартиниса (John Martinis) в университете Калифорнии, Санта Барбара, помимо обширной работы по развитию квантовых алгоритмов \cite{arute2019quantum}, проводились исследования многочастичной локализации \cite{chen2014emulating, roushan2017spectroscopic}, квантового хаоса. В группе проф. Сиддики (Irfan Siddiqi), университет Калифорнии, Бёркли, проводятся эксперименты по наблюдению и изучению отдельных квантовых траекторий и квантовых скачков, которые испытывает кубит под воздействием сильных или слабых измерений (см. \cite{hacohen2016quantum} и ссылки там же). Проф. Астафьев, работавший в Японии и затем в Англии, провёл первые эксперименты по взаимодействию свободного излучения в волноводах с джозефсоновскими кубитами \cite{astafiev2010resonance}. В Дельфтском университете под руководством проф. Ди Карло (Leonardo Di Carlo) проводились одни из первых экспериментов по реализации квантовых алгоритмов на двухкубитных схемах, а затем изучалась возможность цифрового моделирования произвольных гамильтонианов \cite{langford2017experimentally}. Наконец, в группе проф. Волрафа (Andreas Wallraff) проводятся эксперименты с одиночными микроволновыми фотонами, например, по созданию источника последовательности перепутанных друг с другом фотонов \cite{besse2020realizing} или использовать одиночные летящие фотоны для перепутывания удаленных кубитов \cite{kurpiers2018deterministic}.

Как можно видеть, с течением времени обнаружился обширный перечень областей применения сверхпроводниковых квантовых систем, без которых все приведенные выше эксперименты были бы невозможны. Исследования, описанные в данной диссертации, посвящены экспериментальной реализации взаимодействия излучения и вещества в квантовом режиме при помощи сверхпроводниковых искусственных атомов. Данная тема лежит за пределами области цифровых квантовых вычислений и скорее оказывается ближе к аналоговому моделированию одних квантовых систем другими в духе изначального предложения Фейнмана \cite{feynman1982simulating}. Известно, что системы связанных кубитов позволяют экспериментально реализовывать симуляторы спиновых массивов в квантовом режиме, что пытается использовать в своих машинах компания DWave (см, например, \cite{harris2018phase}): однозначное отображение спинового гамильтониана на экспериментальный образец дает надежду на то, что измерение параметров физической системы позволит найти положение минимума энергии в пространстве конфигураций модельного гамильтониана. Подобным образом в ходе мировых исследований было выявлено, что системы связанных многоуровневых трансмонов подходят для аналогового моделирования гамильтониана Бозе-Хаббарда \cite{Orell2019, Ma2019, Hacohen-Gourgy2015, Deng2016, Ye2019, Yan2019}. Такое соответствие открывает целое направление экспериментальных исследований, так как по сути соединяет сверхпроводниковые системы с различными областями теоретической и экспериментальной физики, использующими одну и ту же математическую модель.

\section*{Цель работы}

Целью диссертационной работы является исследование возможности аналогового моделирования взаимодействия излучения и вещества при помощи квантовых сверхпроводниковых устройств, теоретическое и экспериментальное, а также поиск и описание новых эффектов возникающих при таком взаимодействии.

\textbf{Для достижения поставленной цели в ходе исследований были сформулированы следующие задачи:}

\begin{enumerate}
	\item создание экспериментальной базы для исследования сверхпроводниковых систем
	\item измерение однокубитных образцов с целью контроля и улучшения их характеристик при фабрикации
	\item численное моделирование системы двух связанных трансмонов, создание технологических чертежей
	\item экспериментальное исследование образца, изготовленного по созданным чертежам, сопоставление результатов с теоретической моделью
	\item численное моделирование цепочки, состоящей из пяти трансмонов, создание технологических чертежей
	\item экспериментальное исследование образца, сопоставление предсказаний модели и полученных данных
\end{enumerate}


\section*{Методы исследования}

Работа со сверхпроводниковыми квантовыми устройствами требует использования комплекса методик. Основной экспериментальной установкой является рефрижератор растворения, в котором устанавливаются образцы. В нашей лаборатории используется аппарат финнской фирмы BlueFors. Он необходим не столько для обеспечения перехода материалов образца в сверхпроводящее состояние (достижение температуры перехода алюминия в 1.3 К, например, не требует такого типа рефрижераторов), сколько потому, что рабочие частоты переходов системы составляют всего лишь несколько ГГц: для того, чтобы система постоянно находилась в  основном энергетическом состоянии, требуются температуры ниже 100 мК. Рефрижератор должен быть соответствующим образом укомплектован, чтобы к образцу возможно было подключить коаксиальные выводы и осуществлять подачу на него микроволновых сигналов. В нашей лаборатории для этих целей используются оригинальные системы держателей образцов, изготовленных из бескислородной меди и немагнитные кабельные сборки. Помимо этого, требуется обеспечить магнитное экранирование образцов магнитомягким материалам с высокой магнитной проницаемостью.

Сами образцы обычно представляют собой кремниевые кристаллы, на которых напылен методом электронно-лучевого осаждения тонкий слой алюминия. Структуры в металле создаются при помощи фото или электронной литографии, в зависимости от требуемого размера элементов. Например, джозефсоновские переходы формируются на резистивной маске электронным лучом, а резонаторы и конденсаторы кубитов методами фотолитографии. Далее происходит проявление и, например, травление в плазме металла через образовавшиеся окна. В целом, изготовление образцов -- это сложный многоступенчатый процесс с большим числом вариаций процессов и комбинаций используемых материалов, не ограничивающихся, конечно, лишь алюминием и кремнием. Автор не занимался производством образцов в рамках данной диссертационной работы, поэтому более подробное описание всех технологических процессов здесь приводиться не будет.

Измерение образцов производится при помощи коммерческого сверхвысокочастотного оборудования. Одним из главных элементов являются малошумящие усилители на двумерном электронном газе, обладающие минимальным добавленным шумом порядка 1.5 К в широкой полосе частот. Такие устройства позволяют регистрировать сигналы на уровне одного фотона по мощности используя разумное число усреднений (производитель -- шведская фирма Low Noise Factory). Превосходят их по этому параметру только джозефсоновские параметрические усилители, добавленный шум которых примерно в 10 раз меньше и лимитируется уже квантовыми флуктуациями электромагнитного поля на входе. Однако, это гораздо более редкие устройства, которые пока что коммерчески недоступны и изготовляются в каждой лаборатории самостоятельно. Другими приборами, использующимися в эксперименте, являются векторные анализаторы цепей, микроволновые генераторы СВЧ, спектральные анализаторы (фирмы-производители немецкая Rohde \& Schwartz и американская Keysight). Помимо СВЧ устройств, работающих с непрерывными сигналами, для получения СВЧ импульсов применяется квадратурная модуляция с одной боковой полосой и подавлением несущей при помочи ВЧ устройств, генерирующих и снимающих сигналы на промежуточной частоте. Это также коммерческие цифровые приборы, с частотой дискретизации как минимум 1 ГВыб/с и аналоговой полосой 1 ГГц (производители немецкая Spectrum, американская Keysight). Работа с оборудованием осуществляется при помощи программного кода на языке Python, код находится в открытом доступе.

\section*{Основные положения, выносимые на защиту}

\begin{enumerate}
	\item Разработана программно-инструментальная база для работы со сверхпроводниковыми квантовыми устройствами с использованием автоматизации измерений при помощи методов компьютерного зрения.
	\item Разработана и исследована экспериментально система из двух связанных трансмонов, построена квантовомеханическая модель, объясняющая наблюдаемые спектральные линии и предсказывающая эффекты взаимодействия системы и падающего на нее излучения
	\item Разработана и экспериментально исследована цепочка из пяти трансмонов, моделирующая фотонный транспорт через гамильтониан Бозе-Хаббарда, связанный с резервуарами на его краях; построена теоретическая модель, позволяющая численно рассчитать неравновесную динамику с учетом диссипации и внешнего вынуждения и предсказывающая спектральные свойства системы; показан переход от классического линейного режима к нелинейному, квантовому режиму работы системы с увеличением мощности падающего излучения
\end{enumerate}
	
\section*{Научная новизна исследований}

\begin{enumerate}
	\item Впервые были применены методы машинного зрения к задаче автоматизации измерений сверхпроводниковых квантовых устройств, что позволило проводить полный цикл экспериментов с однокубитными образцами без участия оператора
	\item Впервые проведен полный и подробный анализ спектральных переходов в системе двух связанных трансмонов, впервые были обнаружены необычные проявления гибридизации излучения и вещества в составной системе, значительно изменяющие структуру энергетических уровней системы
	\item Впервые был продемонстрирован транспорт фотонов в квантовом режиме через цепочку из пяти связанных трансмонов, впервые показан переход от классического режима транспорта к квантовому режиму фотонной блокады; продемонстрированы многофотонные переходы на многочастичные возбужденные состояния, точно предсказанные теоретическим расчетом.
\end{enumerate}

\section*{Практическая значимость работы}

Исследование сложных квантовых систем представляет огромную значимость для современной науки и техники. Находясь на переднем крае физики, исследования по созданию и контролю многочастичных квантовых систем могут в конечном итоге привести к созданию практически полезных макроскопических квантовых устройств, которые найдут применение в решении задач материаловедения, машинного обучения, оптимизации и криптографии. Принципиальная сложность численного расчета квантовой системы, выражающаяся в экспоненциальном росте гильбертова пространства состояний с увеличением числа её подсистем, является с одной стороны проблемой, а с другой стороны -- возможностью, которую современной науке требуется использовать в своих целях. Эксперименты, проведенные автором в рамках данной диссертации подтверждают, что контроль небольших по размеру систем возможен, и применение их для моделирования известных гамильтонианов действительно имеет перспективы. Исследования и методики, описанные в диссертации, будут служить основой для дальнейшего увеличения числа кубитов на образце, совершенствования архитектуры чипов, масштабирования измерительных установок и поиска новых применений и задач, для которых возможно будет использовать сверхпроводниковые квантовые технологии.

\section*{Личный вклад автора}

Основные результаты, изложенные в данной диссертации, получены лично автором. Соискатель проводил численные расчёты и экспериментальные исследования образцов, также автор принимал участие в
их изготовлении и анализировал полученные результаты. Во всех случаях
заимствования материалов других авторов в диссертации приведены ссылки
на литературные источники.

\section*{Аппробация работы}

Основные теоретические и практические результаты диссертации
опубликованы в \textbf{5} статьях в научных журналах, входящих в перечень ВАК
РФ.


\chapter{Проектирование квантовых сверхпроводниковых устройств}

\section{Квантование электрических цепей}


Процесс создания образца начинается прежде всего с расчета его характеристик. Для определения энергетической структуры системы применяется стандартная процедура квантования, берущая начало в классической механике: сначала составляется лагранжиан, который через преобразование Лежандра трансформируется в Гамильтониан, содержащий только обобщенные координаты и импульсы, и, наконец, производится переход к операторному представлению. 

\subsection{Классические уравнения движения}


Проведем эту процедуру сначала для простейшей электрической цепи, представляющей параллельно соединенные катушку с индуктивностью $L$ и конденсатором емкости $C$. Известно, что напряжение $U_L$ на катушке пропорционально скорости изменения протекающего через нее тока $I_L(t)$: $U_L(t) = - L \dot I_L(t)$, а напряжение $U_C(t)$ на конденсаторе пропорционально заряду $Q$ на нем: $U_C(t) = Q(t)/C$. Из сохранения заряда следует, что $I_L(t) = I_C(t) \equiv \dot Q(t)$. Приравнивая оба напряжения, мы получаем уравнение движения:

\begin{equation}
L \ddot Q + Q/C = 0.
\end{equation}

Подставляя в это уравнение заряд, гармонически зависящий от времени, $Q(t) = Q^0 \cos(\omega_r t)$, мы получаем стандартное выражение для собственной (резонансной) частоты $LC$ осциллятора:

\begin{equation}
	\omega_r = \sqrt{1/LC}.
\end{equation}

Теперь введем вместо заряда другую переменную, которая для элемента цепи будет рассчитываться как интеграл по времени от напряжения, возникающего на нем:
\begin{equation}
	\Phi(t) = \int_{-\infty}^{t} U(\tau) \diff \tau.
\end{equation}
Напряжение на конденсаторе теперь может быть математически выражено как производная по времени от этой величины, причем в данном случае она не имеет физического смысла. Напротив, для катушки $\Phi(t) = \int_{-\infty}^{t} L \dot I_L(\tau) \diff \tau = LI_L(t)	$  легко вычисляется и совпадает по смыслу с магнитным потоком, проходящем через нее, откуда и проистекает выбор обозначения данной величины. Теперь запишем для тока через конденсатор: $I_C(t) = \dot Q_C(t) = C \dot U_C(t)$, откуда с использованием равенства $U_C(t) = U_L(t)$ получаем альтернативное уравнение движения уже на магнитный поток:

\begin{equation}
	C \ddot \Phi(t) + \Phi(t)/L = 0.
\end{equation}
Подставляя в это уравнение гармоническую временную зависимость, мы получим ту же самую резонансную частоту, что и ранее для заряда.

Наличие двух эквивалентных уравнений движения, записанных в разных координатах, означает возможность составления, соответственно, двух эквивалентных лагранжианов:
\begin{align}
	\mathcal{L}_Q &= \frac{Q^2}{2 C} - \frac{L \dot Q^2}{2},\\
	\mathcal{L}_\Phi &= \frac{\Phi^2}{2 L} - \frac{\dot \Phi^2}{2C}.
\end{align}
Напомним, что такие лагранжианы могут быть выбраны, поскольку позволяют получить требуемые уравнения движения при решении соответствующей вариационной задачи с помощью уравнения Эйлера-Лагранжа
\begin{equation}
\frac{d}{d t}\frac{\partial \mathcal{L}_X}{\partial \dot X} - \frac{\partial \mathcal{L}_X}{\partial X} = 0,
\end{equation}
где $X = Q, \Phi$. Читатель может заметить, что выбор из двух лагранжианов представляет собой ни что иное, как выбор ролей емкостной и индуктивной энергий в качестве кинетической либо потенциальной. 

Для понимания полезно также рассмотреть, как альтернативное описание может быть создано для привычной механической системы, например, для шарика массы $m$ на пружинке с коэффициентом жесткости $k$. Второй закон Ньютона для системы записывается как
\begin{equation}
	m \ddot x(t) + k x(t) = 0,
\end{equation}
где $x(t)$ обозначает смещение шарика относительно положения равновесия. Теперь введем новую переменную
\begin{equation}
	p(t) = \int_{-\infty}^{t} - k x(\tau) \diff \tau,
\end{equation}
имеющую физический смысл импульса, накопленного шариком под действием силы со стороны пружины к моменту времени $ t $. Для интересующего нас колебательного движения шарика этот импульс всегда ограничен и колеблется вокруг нуля. Скорость шарика будет выражаться как $v(t) = p(t)/m$, а уравнение движения станет выглядеть как (с учетом того, что $ x(t) = - \dot p(t) / k $)
\begin{equation}
	\ddot p(t)/k + p(t)/m = 0.
\end{equation}

Как видим, имеется полная взаимозаменяемость двух описаний в координатах или импульсах, причем лагранжианы выглядят следующим образом:
\begin{align}
\mathcal{L}_x &= \frac{m \dot x^2}{2} - \frac{k x^2}{2},\\
\mathcal{L}_p &= \frac{\dot p^2}{2 k} - \frac{p^2}{2 m}.\label{eq:Lp}
\end{align}

Показанный дуализм связан как с понятием канонически сопряженных координат и каноническими преобразованиями, так и с теорией представлений и принципом неопределенности в квантовой механике.

\subsection{Преобразование Лежандра и гамильтониан}

Концепция импульса в теоретической механике переходит к обобщенному импульсу, определяемому как 
\begin{equation}
	p_i = \frac{\partial \mathcal{L}_x}{\partial \dot x_i},
\end{equation}
где $ \mathcal{L}_x $ - это Лагранжиан системы, записанный через набор обобщенных координат $\{x_i\}$. В уравнении Эйлера-Лагранжа можно видеть корень такого определения: временная производная от определенной таким образом величины оказывается приравнена производной Лагранжиана по обобщенной координате. Сравнивая эту форму со вторым законом Ньютона, можно ввести и понятие обобщенной силы, как градиента потенциальной энергии в пространстве $q_i$. Аналогично, если лагранжиан записан в форме \eqref{eq:Lp}, то обобщенный импульс будет определен как
\begin{equation}
x_i = \frac{\partial \mathcal{L}_p}{\partial \dot p_i}.
\end{equation}

Как видим, математический аппарат теоретической механики вполне допускает, что обобщенным импульсом может оказаться пространственная координата. Общая теория подобного рода канонических преобразований изложена, например, в \cite{shmutzer1976}.

Часто в аналитической механике используется координатный базис, в котором обобщенные импульсы связаны именно с физическим движением тел. Однако для электрических же систем лагранжиан может быть с равной убедительностью записан как в потоковом базисе, так и в зарядовом базисе, и невозможно отдать предпочтение ни одному из двух вариантов. К примеру, в базисе зарядов ``обобщенный импульс'' записывается как $\Phi = \partial \mathcal{L}_Q / \partial \dot Q$, причем здесь для него опять выбрано обозначение магнитного потока. Для гармонического осциллятора
\begin{equation}
	\Phi = L\dot Q = LI_L. \label{eq:Phi}
\end{equation}

Когда обобщенный импульс выбран, для получения гамильтониана используется преобразование Лежандра. Приведем его для механической системы в координатном базисе:
\begin{equation}
	\mathcal{H} = \sum_i p_i \dot x_i - \mathcal{L}_x.
\end{equation}
Проводя соответствующее преобразование для электрического осциллятора в базисе зарядов получим, подставляя $\dot Q$, выраженное через $\Phi$:
\begin{equation}
	\mathcal{H} = \frac{\Phi^2}{2 L} + \frac{Q^2}{2C}.\label{eq:planck_electric_osc}
\end{equation}
Отметим, что гамильтониан содержит уже обе канонически сопряженные переменные и не зависит от выбора первоначальной координаты так, как ранее зависел лагранжиан.

\subsection{Матричная механика Гейзенберга}

С 1900 по 1925 год единственной теорией, позволявшей описывать линейчатые спектры простейших одноэлектронных атомов была так назваемая ``старая квантовая механика'', так или иначе использовавшая модель Бора-Зоммерфельда, предполагающая квантование интеграла. В 1925 году Вернер Гейзенберг, Макс Борн и Паскуаль Йордан создали матричную квантовую механику, которая без изменений существует и используется до сих пор в нерелятивистском пределе; более широкое распространение, однако, сейчас имеет эквивалентная теория -- волновая механика Эрвина Шредингера, опубликованная им в 1926 году. Матричная механика впервые сводит задачу определения экспериментальных спектров квантовых систем к задаче поиска собственных значений матрицы, которой заменяется оператор Гамильтона. 

Для того, чтобы увидеть исток матричной механики, прежде всего полезно вспомнить правило Бора-Зоммерфельда из старой квантовой механики:
\begin{equation}
	\oint p_i \diff x_i = n_i \hbar, \label{eq:bohr-sommerfeld}
\end{equation}
определяющее дискретный набор классических траекторий, доступных системе. Это правило связано с классическим адиабатическим (квазистатическим) инвариантом $I = \oint p_i \diff x_i$, берущемся по одному периоду движения системы, и, фактически, указывает на квантование именно его значений. 

Открытие правила Бора-Зоммерфельда было предопределено установлением знаменитой формулы $E = \hbar \omega$, впервые выдвинутой Максом Планком в 1900 году для решения ультрафиолетовой катастрофы и примененной также Альбертом Эйнштейном в 1905 для свободного излучения в фотоэффекте. В 1911 году на Солвеевской конференции Хендриком Лоренцом был задан вопрос: как же может зависеть энергия осциллятора от его частоты в классической механике? Ведь для классических осцилляций частота и энергия не связаны напрямую. Также было неясно, как энергия квантового маятника может измениться при очень медленном изменении длины его подвеса, которое заведомо не может вызвать переходы между энергетическими уровнями системы из-за отсутствия необходимых для этого частотных компонент. Именно этот вопрос и привел к понятию адиабатического инварианта: выяснилось, что для так называемого осциллятора Рэлея-Лоренца при медленном изменении длины подвеса частота и энергия изменяются пропорционально друг другу, а отношение их как раз и равняется $I$. Таким образом выяснилось, как квантовомеханическая формула может быть увязана с классическими понятиями.

Триумфом уравнения \eqref{eq:bohr-sommerfeld} стала модель атома Нильса Бора, предложенная в 1913 году и описавшая, наконец, линейчатую структуру спектра атомов водорода и других одноэлектронных атомов, используя правило квантования адаибатического инварианта по классическим круговым орбитам. Арнольд Зоммерфельд продолжил развитие этой модели, введя эллиптические орбиты и расширив сферу ее правильных предсказаний. Однако, несмотря на это, старая квантовая механика все равно не смогла описать системы с более чем одним электроном. В частности, для атома гелия, представляющего систему трех тел, классические траектории могут быть хаотическими и, следовательно, апериодическими, что принципиально не дает возможности рассчитать адиабатический инвариант \cite{wintgen1992semiclassical}. Исследования в направлении квазиклассического описания классических хаотических систем ведутся до сих пор. Более того, старая квантовая теория не объясняла, как именно происходят переходы между состояниями и как рассчитывать наблюдаемые интенсивности спектральных линий.

Итак, на смену старой квантовой механике, основанной на правиле Бора-Зоммерфельда, пришла матричная механика Гейзенберга. Естественно, основой теории по-прежнему служило понятие о квантовании энергетических уровней и дискретности разрешенных состояний. Однако путь к новому законченному формализму, занявший более 10 лет, был очень труден \cite{van2007sources}, и мы приведем его лишь кратко. Прежде всего, Гейзенберг использовал знание того, что любой процесс с периодом $T$ можно представить в виде ряда Фурье. Например, мы можем записать разложение для координаты механической системы:
\begin{equation}
	x(t) = \sum_{n=-\infty}^{+\infty} x_n e^{2\pi i n t/T}.\label{eq:fourier}
\end{equation}

Следующее соображение состоит в том, что квантовая система может излучать энергию лишь дискретными порциями при переходе с уровня $n$ на уровень $m$, испуская фотон на частоте $(E_n - E_m)/\hbar$. Гейзенберг утверждал, что согласно принципу соответствия, такое излучение в классическом пределе должно определяться компонентой соответствующей частоты в разложении \eqref{eq:fourier}. Иными словами, если излучение на частоте $(E_n - E_m)/\hbar$ присутствует, значит, должно быть и что-то в физической системе, на такой частоте осциллирующее. Такая формулировка при всей своей поверхностности и, казалось бы, натянутости приводит нас к совершенно точному уравнению для эволюции так называемых ``матричных элементов'' $x_{nm}$:
\begin{equation}
	x_{nm}(t) = e^{i (E_n - E_m) t / \hbar} x_{nm}(0).\label{eq:heisenberg_evolution}
\end{equation}
По Гейзенбергу связь матричных элементов и классических величин в том, что в классическом пределе $x_{nm}(t)$ при $n, m \gg 1$ переходит к Фурье-компоненте соответствующей частоты $(E_n - E_m)/\hbar$, выбранной из разложения \eqref{eq:fourier}. 

Посмотрим, как эта логика работает в случае гармонического осциллятора. В классическом пределе, используя формулу Эйлера,
\begin{align}
	x(t) &= \sqrt{E/2k} (e^{-i\omega_r t} + e^{i\omega_r t}),\\
	p(t) &= i\sqrt{E/2m} (e^{-i\omega_r t} - e^{i\omega_r t}),
\end{align}
где $ E $ -- это классическая полная энергия. Можно видеть, что в динамике осциллятора имеются только компоненты на частоте $\pm \omega_r$. Поэтому принцип соответствия требует, чтобы матричные элементы $x_{nm}(0)$ были ненулевыми только для соседних значений индексов $n$ и $m$, а энергетические уровни -- эквидистантными, $E_n = \hbar \omega_r \cdot n$. Значения же этих матричных элементов обуславливаются энергиями состояний, между которыми происходят переходы -- в классическом пределе амплитуда колебаний однозначно связана с их энергией. Таким образом, можно сразу же изобразить, например, $x_{nm}(0)$ осциллятора в матричном виде:
\begin{equation}
	x_{nm}(0) = \sqrt{\frac{2\hbar \omega_r}{k}}\left(	\begin{matrix}
	0 & \sqrt{1} & 0 & 0&  \dots \\
	\sqrt{1} & 0 &\sqrt{2} & 0& \dots \\
	0 & \sqrt{2} & 0 & \sqrt{3} &  \dots\\
	0 & 0 & \sqrt{3} & 0 & \dots \\
	\vdots & \vdots & \vdots & \vdots  & \ddots
	\end{matrix}\right).
\end{equation}

Также легко можно записать матрицу и для импульса $p_{nm}(0)$. Далее, легко проверить, что задание сопряженных переменных таким образом доставит диагональный вид матрице гамильтониана $\mathcal{H}_{nm} = \delta_{nm} \hbar \omega_r n $.

Из установленного матричного вида для координаты и импульса следует также и их некоммутативность:
\begin{equation}
	[\hat x, \hat p] = \hat x \hat p - \hat p \hat x = i\hbar,
\end{equation}
где шапочки над буквами обозначают их матричную природу.

Далее, эволюция во времени матрицы $\hat A$ некоей физической наблюдаемой в гейзенберговском подходе описывается дифференциальным уравнением в матричной форме:
\begin{equation}
	\frac{\diff \hat A}{\diff t} = \frac{i}{\hbar} [\hat{\mathcal{H}}, \hat A], \label{eq:heisenberg_picture}
\end{equation}
которое следует из \eqref{eq:heisenberg_evolution} и диагонального вида матрицы гамильтониана. Отметим здесь поразительное сходство этого уравнения с классическим уравнением Лиувилля на некую функцию обобщенных координат $ A(x, p) $:
\begin{equation}
	\frac{\diff A}{\diff t} = \{\mathcal{H}, A\} \equiv \frac{\partial A}{\partial p} \frac{\partial \mathcal{H}}{\partial p} -  \frac{\partial A}{\partial q}\frac{\partial \mathcal{H}}{\partial q}.
\end{equation}

В принципе, уравнения \eqref{eq:heisenberg_picture} уже достаточно для описания всех изолированных квантовых систем. Единственной проблемой, которую требуется решить, является нахождение матричных элементов в момент времени $t=0$. Для гармонического осциллятора сделать это достаточно просто, однако для более сложных нелинейных систем

\subsection{Волновая механика Шредингера}


Практически одновременно со статьей Гейзенберга, Борна, Йордана австриец Эрвин Шрёдингер впервые вывел своё знаменитое уравнение, основанное на гипотезе Луи Де Бройля о ``волнах материи'' (корпускулярно-волновом дуализме), выдвинутой им в своей докторской работе в 1924 году. Интересен тот факт, что опыты по дифракции электронов были проведены лишь позднее, и гипотеза, не имея реального экспериментального подтверждения, была основана лишь на аналогии с понятием о корпускулярно-волновых свойствах света, введенными Альбертом Эйнштейном для объяснения фотоэффекта. Публикация, содержащая уравнение Шредингера и названная ``Квантование как задача о собственных значениях'', увидела свет лишь в начале 1926 года \cite{schrodinger2003collected}, уже после работы о матричной механике. Вот что писал об этой статье Гейзенберг в письме к своему ровеснику Вольфгангу Паули в том же году: 

\begin{center}
\begin{minipage}{0.9\textwidth}
\textit{``The more I reflect on the physical portion of Schrödinger's theory the more disgusting I find it. What Schrödinger writes on the visualizability of his theory, I consider trash. The greatest result of his theory is the calculation of the matrix elements.''}
\end{minipage}
\end{center}
Как видим, возникла некоторая соревновательность между двумя подходами к описанию квантовых систем. Однако история показала, что на самом деле два подхода эквивалентны. Уравнение Шрёдингера оказалось более удобно для расчета спектров квантовых систем, а гейзенберговские уравнения движения чаще используются в квантовой оптике для описания динамики наблюдаемых величин в простых системах.

Итак, гипотеза Де Бройля состояла в том, что частица на самом деле может проявлять свойства волны с волновым вектором $\mathbf{k} = \mathbf{p}/\hbar$ и частотой $\omega = E/\hbar$. Значит, математически можно записать для амплитуды этой волны следующее выражение:
\begin{equation}
	\psi (\mathbf{x},t) = \psi_0 e^{i(\omega t - \mathbf{k} \mathbf{x})} = \psi_0 e^{i(E t - \mathbf{p} \mathbf{x})/\hbar}. \label{eq:psi}
\end{equation}
Каким же должно быть уравнение, чтобы его решение не только давало правильное выражение для волны свободной частицы, но и описывало дискретные спектры частиц в произвольном потенциале, да еще и удовлетворяло принципу соответствия, воспроизводя правильные законы движения для систем в классическом пределе?

Несмотря на кажущуюся трудность вопроса, надежда получить из такого подхода правильные результаты была сильна. Например, в задачах, связанных с волнами, дискретные частотные спектры оказываются повсеместным явлением, если рассматриваются нормальные колебательные моды. Напомним, что нормальной модой или стоячей волной называется факторизованное решение волнового уравнения
\begin{equation}
	\frac{\partial^2 u}{\partial x^2} - c^2 \frac{\partial^2 u}{\partial t^2} = 0\label{eq:wave_eq}
\end{equation}
вида
\begin{equation}
	u_n(x, t) = u_n(x) e^{i\omega_n t},
\end{equation}
где ${\omega_n}$ как раз и является набором дискретных собственных частот. Аналогично, если частица проявляет волновые свойства, то и стоячие волны материи с дискретными частотами должны быть возможны.

Шредингер начинает свой вывод с классического уравнения Гамильтона-Якоби на действие $\mathcal{S} = \mathcal S (x,t)$ \cite{schrodinger2003collected}. Мы для простоты запишем его для частицы массой $m$ с одной степенью свободы:
\begin{equation}
	\mathcal{H}\left(x, \frac{\partial \mathcal S}{\partial x}\right) = \frac{\partial \mathcal S}{\partial t} = E, \label{eq:hamilton-jakobi}
\end{equation}
где последнее равенство достигается за счет отсутствия явной зависимости от времени в потенциальной энергии \cite{shmutzer1976}. Далее совершается подстановка
\begin{equation}
	\mathcal S = K \log \psi, \label{eq:psi_action}
\end{equation}
где неизвестная функция $\psi(x,t)$ пока что не имеет определенного физического смысла. Подставляя это выражение в \eqref{eq:hamilton-jakobi}, получим
\begin{equation}
	\left(\frac{\partial \psi}{\partial x} \right)^2 + \frac{2m}{K^2}(U(x) - E)\psi^2 = 0.
\end{equation}
Далее предлагается решить вариационную задачу на $\psi(x, t)$:
\begin{equation}
	\delta J = \delta \int \diff x \left[ \left(\frac{\partial \psi}{\partial x} \right)^2 +  \frac{2m}{K^2}(U(x) - E)\psi^2 \right] = 0.
\end{equation}
Отсюда из уравнения Эйлера-Лагранжа \cite{hilbertcourant} сразу же получаем стационарное уравнение Шредингера уже практически в современном виде:
\begin{equation}
	-\frac{K^2}{2m}\frac{\partial^2 \psi}{\partial x^2} + U(x)\psi = E \psi.\label{eq:shroedinger_stationary}
\end{equation}
Единственным, что требовалось установить, была величина константы $K$, имеющей размерность действия. Естественным образом, для совпадения с экспериментом она должна быть положена равной $\hbar$.

Отметим, что в этом выводе (первая часть работы Шредингера) ничего не говорится о явной форме временной зависимости функции $\psi(x, t)$. Шредингер говорит, что был неправ, когда поначалу называл уравнение \eqref{eq:shroedinger_stationary} волновым. На самом деле, это уравнение амплитудное, или уравнение вибрации. Его главная проблема в том, что оно содержит обязательный параметр $E$ и не содержит производных по времени. Таким образом, оно не может описать неконсервативную систему, взаимодействующую, например, с излучением. Для того, чтобы получить настоящее волновое уравнение Шредингер попытался добавить производные по времени в свое уравнение, используя тот факт, что для гармонического решения $\psi(x,t) \propto \exp[-i E/\hbar t]$ выполнено
\begin{equation}
	\frac{\partial^2 \psi(x,t)}{\partial t^2} = - \frac{E^2}{\hbar^2} \psi(x,t).\label{eq:shroedinger_2nd_time_der}
\end{equation}
Для того, чтобы получить истинное волновое уравнение, не содержащее параметра $E$ и дающее уравнение \eqref{eq:shroedinger_stationary} в случае гармонической зависимости от времени функции $\psi(x,t)$, Шредингер дифференцирует уравнение \eqref{eq:shroedinger_wave} по координате еще два раза, выражает $\partial^2\psi/\partial x^2$ и $E\psi$ через прежнее стационарное уравнение, а затем разрешает всё относительно $E^2\psi$. Получившееся выражение подставляется в правую часть \eqref{eq:shroedinger_2nd_time_der} и получается новое уравнение:
\begin{equation}
	\left( \frac{\hbar^2}{2m}\frac{\partial^2}{\partial x^2} - U(x)\right)^2 \psi + \frac{4}{\partial^2 \psi}{\partial t^2} = 0.\label{eq:shroedinger_wave}
\end{equation}
Это Шредингер и называет однородным и общим волновым уравнением на скалярное поле $\psi$. Однако оно уже не имеет такой простой формы, как \eqref{eq:wave_eq}: здесь мы видим четвертый порядок производной по координате, что походит на задачу о собственных колебаниях твердых тел. Например, для колебаний упругой пластины волновое уравнение действительно содержит два лапласиана \cite{hilbertcourant}.
Далее в предположении независимости $ U(x) $ от времени уравнение \eqref{eq:shroedinger_wave} при обратной подстановке гармонической зависимости от времени для $\psi$, может быть разложено на множители следующим образом:
\begin{equation}
	\left(\frac{\hbar^2}{2m}\frac{\partial^2}{\partial x^2} - U(x) + E \right)\left(\frac{\hbar^2}{2m}\frac{\partial^2}{\partial x^2} - U(x) - E \right)\psi = 0,
\end{equation}
возвращаясь таким образом к исходному стационарному уравнению \eqref{eq:shroedinger_stationary}.  Шредингер в своей статье пишет, что такое разделение не может быть само по себе убедительным в плане строгости, однако часто встречается в решении дифференциальных уравнений в частных производных и обосновывается уже после через полноту базиса собственных функций. Однако, к сожалению, уравнение \eqref{eq:shroedinger_wave} не может быть разделено таким образом в случае зависимости потенциала от времени и, таким образом, не годится для нестационарных задач. Более того, вывод этого уравнения удается строго повторить только если предполагать, что $U(x)$ -- константа (читателю также предлагается проверить это).

Поэтому далее Шредингер отказывается от соблюдения обычной формы волнового уравнения и для удаления свободного параметра $E$ выбирает другую, более простую, подстановку:
\begin{equation}
	\frac{\partial \psi}{\partial t} = \pm \frac{i}{\hbar} E \psi.
\end{equation}
Тогда нестационарное уравнение Шредингера запишется уже в совершенно современном виде:
\begin{equation}
\pm \frac{i}{\hbar}\psi  = -\frac{\hbar^2}{2m} \Delta^2\psi + U(x,t)\psi.
\end{equation}
Именно о такой, и только о такой подстановке Шредингер пишет в своей обзорной статье, опубликованной уже в конце 1926 года и названной ``Колебательная теория механики атомов и молекул'' \cite{schrodinger1926undulatory}.

Несмотря на успешность уравнения Шредингера, физический смысл понятий, лежащих в его основе, в его работах полностью не раскрывается, и классическая интерпретация волн материи как стоячих волн не выдерживает подробного анализа. К примеру, квантование частот стоячих волн в классической механике совершенно не сопряжено с квантованием энергий: амплитуда каждой стоячей волны может быть произвольной. Далее, понятие нормальных колебаний определено только в линейных системах, а в таких системах переходы между различными модами под действием излучения на разности их частот происходить не могут: для этого необходимы нелинейные процессы смешивания. Для нелинейных же систем понятие нормальных мод строго не определено \cite{kerschen2009nonlinear}. В рамках сказанного непротиворечивая интерпретация физического смысла волновой функции как стоячей волны всё-таки невозможна, и, вероятно, именно эти пробелы вызвали отторжение у Гейзенберга. Поэтому прежде, чем переходить к расчету волновых функций для гармонического электрического осциллятора, мы введем современную аксиоматику и будем далее использовать только её.

\subsection{Понятия современной квантовой механики}

На момент создания волнового уравнения для частиц еще не существовало понятия о гильбертовом пространстве векторов состояний, дираковского формализма бра-кет векторов, теории представлений, а также понятия об импульсе как о дифференциальном операторе. Также не существовало и интерпретации квадрата абсолютного значения \textit{волновой функции} $|\psi(\mathbf{x},t)|^2$ как вероятности обнаружить частицу в той или иной точке пространства. Современная квантовая механика, наоборот, строит свою аксиоматику изначально основываясь на этих понятиях и принципе соответствия, а уравнение Шредингера оказывается выводимым следствием. 

Сейчас в университетских программах главный шаг в процедуре квантования состоит в замене обобщенного импульса дифференциальным оператором:
\begin{equation}
	\mathbf p \rightarrow -i\hbar \nabla. \label{eq:p_to_diff}
\end{equation}
Насколько возможно было заметить это соответствие в 1926 году? В принципе, ключевые открытия уже были сделаны. В первую очередь, вспомним важнейшее соотношение из классической теоретической механики:
\begin{equation}
	\mathbf p = \mathcal S.
\end{equation}
С учетом очевидной связи волновой функции и действия через выражение \eqref{eq:psi_action}, а также того, что именно кинетическая энергия дает все члены с производными в стационарном уравнении Шредингера, вывод о замене физической наблюдаемой оператором не заставил себя долго ждать. Уже в августе 1926 года Поль Дирак в докторской статье явно использует замену \eqref{eq:p_to_diff} \cite{dirac1926theory}. Дополнительно принятие перехода к операторному виду было облегчено и матричной механикой Гейзенберга, о выведении которой из уравнения Шредингера, кстати, было сказано Дираком в той же статье.

\chapter{Экспериментальные методы}

\chapter{Взаимодействие двухатомной искусственной молекулы и излучения}

\chapter{Квантовый фотонный транспорт в модели Бозе-Хаббарда}

\chapter{Заключение}


\appendix
\renewcommand*\thesection{\Alph{chapter}.\arabic{section}}


\renewcommand\bibname{Список литературы}
\bibliographystyle{ugost2008}
\bibliography{dissertation.bib}
\end{document}
